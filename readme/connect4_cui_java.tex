% latex uft-8
\documentclass[uplatex,a4paper,11pt,oneside,openany]{jsbook}
%
\usepackage[dvipdfmx]{graphicx}
\usepackage{amsmath,amssymb}
\usepackage{bm}
\usepackage{graphicx}
\usepackage{ascmac}
\usepackage{setspace}
\usepackage{here}
 \usepackage{url}
\usepackage{listings,jlisting} %日本語のコメントアウトをする場合jlistingが必要
%ここからソースコードの表示に関する設定
\usepackage{xcolor,comment}

\definecolor{mygreen}{rgb}{0,0.6,0}
\definecolor{mygray}{rgb}{0.5,0.5,0.5}
\definecolor{mymauve}{rgb}{0.58,0,0.82}

\begin{comment}
\lstset{
  backgroundcolor=\color{white},   % choose the background color; you must add \usepackage{color} or \usepackage{xcolor}; should come as last argument
  basicstyle=\footnotesize,        % the size of the fonts that are used for the code
  breakatwhitespace=false,         % sets if automatic breaks should only happen at whitespace
  breaklines=true,                 % sets automatic line breaking
  captionpos=b,                    % sets the caption-position to bottom
  commentstyle=\color{mygreen},    % comment style
  deletekeywords={...},            % if you want to delete keywords from the given language
  escapeinside={\%*}{*)},          % if you want to add LaTeX within your code
  extendedchars=true,              % lets you use non-ASCII characters; for 8-bits encodings only, does not work with UTF-8
  firstnumber=1000,                % start line enumeration with line 1000
  frame=single,	                   % adds a frame around the code
  keepspaces=true,                 % keeps spaces in text, useful for keeping indentation of code (possibly needs columns=flexible)
  keywordstyle=\color{blue},       % keyword style
  language=Octave,                 % the language of the code
  morekeywords={*,...},            % if you want to add more keywords to the set
  numbers=left,                    % where to put the line-numbers; possible values are (none, left, right)
  numbersep=5pt,                   % how far the line-numbers are from the code
  numberstyle=\tiny\color{mygray}, % the style that is used for the line-numbers
  rulecolor=\color{black},         % if not set, the frame-color may be changed on line-breaks within not-black text (e.g. comments (green here))
  showspaces=false,                % show spaces everywhere adding particular underscores; it overrides 'showstringspaces'
  showstringspaces=false,          % underline spaces within strings only
  showtabs=false,                  % show tabs within strings adding particular underscores
  stepnumber=2,                    % the step between two line-numbers. If it's 1, each line will be numbered
  stringstyle=\color{mymauve},     % string literal style
  tabsize=2,	                   % sets default tabsize to 2 spaces
  title=\lstname                   % show the filename of files included with \lstinputlisting; also try caption instead of title
}
\end{comment}

%ここからソースコードの表示に関する設定

\lstdefinestyle{customc}{
  belowcaptionskip=1\baselineskip,
  breaklines=true,
  numbers=left,
  frame=single,
  xleftmargin=\parindent,
  language=C,
  showstringspaces=false,
  basicstyle=\footnotesize\ttfamily,
  keywordstyle=\bfseries\color{green!40!black},
  commentstyle=\itshape\color{purple!40!black},
  identifierstyle=\color{blue},
  stringstyle=\color{orange},
}

\lstdefinestyle{customjava}{
  belowcaptionskip=1\baselineskip,
  breaklines=true,
  numbers=left,
  frame=single,
  xleftmargin=\parindent,
  language=java,
  showstringspaces=false,
  basicstyle=\footnotesize\ttfamily,
  keywordstyle=\bfseries\color{green!40!black},
  commentstyle=\itshape\color{purple!40!black},
  identifierstyle=\color{blue},
  stringstyle=\color{orange},
}

\lstdefinestyle{customasm}{
  belowcaptionskip=1\baselineskip,
  frame=L,
  xleftmargin=\parindent,
  language=[x86masm]Assembler,
  basicstyle=\footnotesize\ttfamily,
  commentstyle=\itshape\color{purple!40!black},
}

\lstset{escapechar=@,style=customjava}

\makeatletter
\def\ps@plainfoot{%
  \let\@mkboth\@gobbletwo
  \let\@oddhead\@empty
  \def\@oddfoot{\normalfont\hfil-- \thepage\ --\hfil}%
  \let\@evenhead\@empty
  \let\@evenfoot\@oddfoot}
  \let\ps@plain\ps@plainfoot
\renewcommand{\chapter}{%
  \if@openright\cleardoublepage\else\clearpage\fi
  \global\@topnum\z@
  \secdef\@chapter\@schapter}
\makeatother
%
\newcommand{\maru}[1]{{\ooalign{%
\hfil\hbox{$\bigcirc$}\hfil\crcr%
\hfil\hbox{#1}\hfil}}}
%
\setlength{\textwidth}{\fullwidth}
\setlength{\textheight}{40\baselineskip}
\addtolength{\textheight}{\topskip}
\setlength{\voffset}{-0.55in}
%
\begin{document}
% START DOCUMENT
%
% COVER
\begin{center}
  \huge \par
  \vspace{15mm}
  \huge \par
  \vspace{15mm}
  \LARGE 連珠 (CUI - java) \par
  \vspace{100mm}
  \Large \date \par
  \vspace{15mm}
  \Large \par
  \vspace{10mm}
  \Large S.Matoike \par
  \vspace{10mm}
\end{center}
\thispagestyle{empty}
\clearpage
\addtocounter{page}{-1}
\newpage
\setcounter{tocdepth}{3}
%
\tableofcontents
%

\chapter{はじめに}

これはjava による 「n目並べ」です\\

javaによる「3目並べ」(\url{https://github.com/smat1957/tictactoe_cui_java})
と\\
クラスの構成など、同じ方針で作成しています

また、「3目並べ」で作成した MiniMax Algorithmを必要に応じて参照して下さい

\section{連珠(Connect Four)}

\begin{spacing}{0.6}
  \begin{verbatim}
スタート! [ 5目並べ ]

     0 1 2 3 4 5 6 7 8 9
 0:  . . . . . . . . . .
 1:  . . . . . . . . . .
 2:  . . . . . . . . . .
 3:  . . . . . . . . . .
 4:  . . . . . . . . . .
 5:  . . . . . . . . . .
 6:  . . . . . . . . . .
 7:  . . . . . . . . . .
 8:  . . . . . . . . . .
 9:  . . . . . . . . . .

石を置く場所 xy を指定( あなた の番です ):44

     0 1 2 3 4 5 6 7 8 9
 0:  . . . . . . . . . .
 1:  . . . . . . . . . .
 2:  . . . . . . . . . .
 3:  . . . . . . . . . .
 4:  . . . . ● . . . . .
 5:  . . . . . . . . . .
 6:  . . . . . . . . . .
 7:  . . . . . . . . . .
 8:  . . . . . . . . . .
 9:  . . . . . . . . . .

石を置く場所 xy を指定( あなた の番です ):43
(*** 中略 ***)
石を置く場所 xy を指定( あなた の番です ):53

     0 1 2 3 4 5 6 7 8 9
 0:  . . . . . . . . . .
 1:  . . . . . . . . . .
 2:  . . . . . . . . . .
 3:  . . ● ● ● ● . . . .
 4:  . . . ○ ● . . . . .
 5:  . . . ○ ○ ○ . . . .
 6:  . . . . . ○ . . . .
 7:  . . . . . . ○ . . .
 8:  . . . . . . . ● . .
 9:  . . . . . . . . . .

石を置く場所 xy を指定( あなた の番です ):36

     0 1 2 3 4 5 6 7 8 9
 0:  . . . . . . . . . .
 1:  . . . . . . . . . .
 2:  . . . . . . . . . .
 3:  . . ● ● ● ● ● . . .
 4:  . . . ○ ● . . . . .
 5:  . . . ○ ○ ○ . . . .
 6:  . . . . . ○ . . . .
 7:  . . . . . . ○ . . .
 8:  . . . . . . . ● . .
 9:  . . . . . . . . . .

'X' の勝ち	またね!
\end{verbatim}
\end{spacing}


\subsection{定数定義クラスと主処理}

\begin{lstlisting}[caption=定数定義クラス:N目並べ,label=prog22]
package nmoku;

public final class Constants {
    private Constants() {}
    final static int WIDTH = 7;
    final static int NMOKU = 5;
    final static int NxN = NMOKU * NMOKU;
    final static int WHITE = 1;
    final static int BLACK = -1;
    final static int MARU = WHITE;
    final static int BATSU = BLACK;
    final static int MAX = WHITE;
    final static int MIN = BLACK;
    final static int EMPTY = 0;
    final static int NEXT = 200;
    final static int DRAW = 100;
    final static int PROMPT = 1000;
    final static int RANDOM = 1010;
    final static int MINIMAX = 1001;
    final static int ALPHABETA = 1002;
    final static int MONTECARLO = 1003;
}
\end{lstlisting}

\begin{lstlisting}[caption=主処理:N目並べ,label=prog23]
package nmoku;

public class NMoku {
    public static void main(String[] args) {
        boolean reverse = false;
        if (args.length > 0) {
            if (args[0].equals("-r")) {
                reverse = true;
            }
        }
        Game g = new Game(reverse);
        g.Start();
    }
}
\end{lstlisting}

\subsection{ゲームクラス}

\begin{lstlisting}[caption=ゲームクラス:N目並べ,label=prog24]
package nmoku;

import static nmoku.Constants.*;

public class Game {

    private Player[] player = null;
    private Board board = null;
    private int current_player = 0;
    private Player currPlayer = null;

    public Game(boolean reverse) {
        player = new Player[2];
        if (reverse) {
            player[0] = new Player(MARU, "PC", ALPHABETA);
            player[1] = new Player(BATSU, "あなた", PROMPT);
        } else {
            player[0] = new Player(BATSU, "あなた", PROMPT);
            player[1] = new Player(MARU, "PC", ALPHABETA);
        }
        currPlayer = player[0];
        board = new Board();
    }
    void Start() {
        int winner = NEXT;
        System.out.println("スタート! [ " + NMOKU + "目並べ ]");
        do {
            board.print();
            currPlayer = player[current_player];
            currPlayer.putStone(board);
            winner = board.check();
            current_player = ++current_player % 2;
        } while (winner == NEXT);
        board.print();
        result(winner);
    }
    private void result(int winner) {
        // 結果の表示
    }
}
\end{lstlisting}

\begin{lstlisting}[caption=結果の表示:ゲームクラス内:N目並べ,label=prog25]
private void result(int winner) {
    // 結果の表示
    System.out.println();
    switch (winner) {
        case DRAW:
            System.out.print("引き分け\t");
            break;
        case MARU:
            System.out.print(currPlayer.getName() + "'O' の勝ち\t");
            break;
        case BATSU:
            System.out.print(currPlayer.getName() + "'X' の勝ち\t");
            break;
    }
    System.out.println("またね!");
}
\end{lstlisting}
\
\subsection{石クラス}

\begin{lstlisting}[caption=石クラス:N目並べ,label=prog26]
package nmoku;

import static nmoku.Constants.*;

public class Stone {

    private int locate = 0;
    private int color = EMPTY;

    public Stone(int n, int i) {
        locate = n;
        color = i;
    }
    int getColor() {
        return color;
    }
    void setColor(int i) {
        color = i;
    }
    int getLocate() {
        return locate;
    }
}
\end{lstlisting}

\subsection{盤面クラス}

\begin{lstlisting}[caption=盤面クラス:N目並べ,label=prog27]
package nmoku;

import static nmoku.Constants.*;

public class Board {

    private static Stone[] board = new Stone[WIDTH * WIDTH];
    private static Stone recentStone = null;

    public Board() {
        for (int i = 0; i < WIDTH * WIDTH; i++) {
            board[i] = new Stone(i, EMPTY);
        }
    }
    void setBoard(Stone s) {
        recentStone = s;
        board[s.getLocate()] = s;
    }
    void setEmpty(int xy){
        board[xy]=new Stone(xy, EMPTY);
    }
    boolean canPut(Stone s) {
        if (board[s.getLocate()].getColor() != EMPTY) {
            return false;
        }
        return true;
    }
    int getz(int x, int y) {
        return y * WIDTH + x;	// 0<= x <WIDTH, 0<= y <WIDTH
    }
    int check() {
        return boardScan(recentStone.getLocate() % WIDTH,
                         recentStone.getLocate() / WIDTH);
    }
    void print() {
        // 盤面の表示印刷
    }
    private int boardScan(int x, int y) {
        //盤面の調査(5個並んだかの調査)
    }
    private int boardScanSub(int x, int y, int move_x, int move_y) {
        // 盤面調査の下請け
    }
    boolean isWin(){
        if(check()!=NEXT)return true;
        return false;
    }
    boolean isDraw(){
        return false;
    }
    float evaluate(Player player, int turn){
        if(isWin()){
            if(turn==player.getColor())return (float)(-1.0);
            else if(turn!=player.getColor())return (float)(1.0);
        }
        return (float)0.0;
    }
}
\end{lstlisting}

\begin{lstlisting}[caption=盤面の表示印刷:盤面クラス内:N目並べ,label=prog28]
void print() {
    // 盤面の表示印刷
    final String[] str = {".", "●", "○"};

    System.out.printf("\n    ");
    for (int x = 0; x < WIDTH; x++) {
        if (x < 10) {
            System.out.printf("%2d", x);
        } else {
            System.out.printf("%2c", (char) ('a' - 10 + x));
        }
    }
    System.out.printf("\n");
    for (int y = 0; y < WIDTH; y++) {
        if (y < 10) {
            System.out.printf("%2d: ", y);
        } else {
            System.out.printf("%2c: ", (char) ('a' - 10 + y));
        }
        for (int x = 0; x < WIDTH; x++) {
            int cl = board[getz(x, y)].getColor();
            int num=0;
            if(cl==WHITE)num=2;
            if(cl==BLACK)num=1;
            System.out.printf("%2s", str[num]);
        }
        System.out.printf("\n");
    }
}
\end{lstlisting}

\begin{lstlisting}[caption=盤面の調査:盤面クラス内:N目並べ,label=prog29]
private int boardScan(int x, int y) {
    //盤面の調査(5個並んだかの調査)
    int[] n = new int[4];     //8方向(直線4本分)に並んだ数
    //[\]方向
    n[0] = boardScanSub(x, y, 1, 1);
    //[│]方向
    n[1] = boardScanSub(x, y, 0, 1);
    //[─]方向
    n[2] = boardScanSub(x, y, 1, 0);
    //[/]方向
    n[3] = boardScanSub(x, y, -1, 1);
    for (int i = 0; i < 4; i++) {
        if (n[i] == NMOKU) {
            return recentStone.getColor();
        }
    }
    return NEXT;
}
private int boardScanSub(int x, int y, int move_x, int move_y) {
    // 盤面調査の下請け
    int n = 1;          //置いた場所の1個分で初期化
    int i;
    for (i = 1; i < NMOKU; i++) {
        int z = getz(x + (move_x * i), y + (move_y * i));
        if (!(0 <= z && z < WIDTH * WIDTH))continue;
        if (board[z].getColor() == recentStone.getColor()) {
            n += 1;
        } else {
            break;
        }
    }
    for (i = 1; i < NMOKU; i++) {
        int z = getz(x + (-1 * move_x * i), y + (-1 * move_y * i));
        if (!(0 <= z && z < WIDTH * WIDTH))continue;
        if (board[z].getColor() == recentStone.getColor()) {
            n += 1;
        } else {
            break;
        }
    }
    return n;
}
\end{lstlisting}

\subsection{プレイヤークラス}

\begin{lstlisting}[caption=プレイヤークラス:N目並べ,label=prog30]
package nmoku;

import static nmoku.Constants.*;

public class Player extends Strategy {

      private String Name = null;
      private int Senryaku = 0;
      private int Color = EMPTY;

      public Player(int i, String n, int s) {
          Color = i;
          Name = n;
          Senryaku = s;
      }
      void putStone(Board board) {
          Stone s = null;
          do {
              int te = Te(board);
              s = new Stone(te, Color);
              if (board.canPut(s)) {
                  break;
              }
          } while (true);
          board.setBoard(s);
      }
      private int Te(Board board) {
          int v = 0;
          switch (Senryaku) {
              case PROMPT:
                  v = prompt(Name);
                  break;
              case RANDOM:
                  v = random(Name);
                  break;
              case MINIMAX:
                  v = bestMoveMM(board, this);
                  break;
              case ALPHABETA:
                  v = bestMoveAB(board, this);
                  break;
              case MONTECARLO:
                  break;
          }
          return v;
      }
      int getColor(){
          return Color;
      }
      String getName(){
          return Name;
      }
}
\end{lstlisting}

\subsection{戦略クラス(AlphaBeta)}

盤面の大きさが大きければ大きいほど、またMINIMAXの探索深さが深くなるほど、コンピュータの手を見つけ出すのに時間がかかってしまい、
人間との対戦の場面では現実的ではありません。そこでMINIMAXを改良した、AlphaBetaという戦略を使います。

コンピュータが実用的な対戦相手となるためには、盤面サイズを小さくしたり、探索の深さを妥協したりというチューニングが必要になります。

\begin{lstlisting}[caption=戦略クラス:N目並べ,label=prog31]
package nmoku;

import static nmoku.Constants.*;
import java.io.BufferedReader;
import java.io.InputStreamReader;

public class Strategy {

    private boolean numP(char x) {
        if (('0' <= x) && (x <= '9')) {
            return true;
        }
        return false;
    }
    private int fromA(char x) {
        return (int) (x - 'a');
    }
    int prompt(String name) {
        // 標準入力から
    }
    int random(String name) {
        System.out.printf("。。。 %s 思考中 。。。", name);
        return (int) (Math.random() * (WIDTH * WIDTH));
    }
    private float minimax(Board board, int depth, int turn, Player player) {
        // 三目並べ(TicTacToe)の時と同様に
    }
    int bestMoveMM(Board board, Player player) {
        // 三目並べ(TicTacToe)の時と同様に
    }
    private float alphabeta(Board board, int depth, int turn, Player player, float alpha, float beta) {

        if (board.isWin() || board.isDraw() || depth == 0) {
            float score = board.evaluate(player, turn);
            //System.out.println("evaluated score="+score+", depth="+depth);
            return score;
        }

        for (int i = 0; i < WIDTH * WIDTH; i++) {
            Stone s = new Stone(i, turn);
            if (board.canPut(s)) {

                //board.print();
                //if(turn==MARU)System.out.println("Turn=O");
                //else System.out.println("Turn=X");

                board.setBoard(s);
                float score = alphabeta(board, depth - 1, -turn, player, alpha, beta);
                board.setEmpty(i);
                //System.out.println("k1="+i+", score="+score);

                if (turn == MAX) {
                    alpha = Float.max(score, alpha);
                    if( beta <= alpha )break;
                } else {
                    beta = Float.min(score, beta);
                    if( beta <= alpha )break;
                }
            }
        }
        if(turn==MAX)return alpha;
        return beta;
    }
    int bestMoveAB(Board board, Player player) {
        float bestEval = (float) Integer.MIN_VALUE;
        float alpha = (float) Integer.MIN_VALUE;
        float beta = (float) Integer.MAX_VALUE;
        int bestMove = -1;
        for (int i = 0; i < WIDTH * WIDTH; i++) {
            Stone s = new Stone(i, player.getColor());
            if (board.canPut(s)) {
                //board.print();
                board.setBoard(s);  //BLACKに打たせるときは、-player.getColor()の負号をとり、
                float eval = alphabeta(board, 6, -player.getColor(), player, alpha, beta);
                board.setEmpty(i);  //Board.evaluate で -1 と 1 を入れ替える
                //System.out.println("k="+i+", eval="+eval);
                if (eval > bestEval) {
                    bestEval = eval;
                    bestMove = i;
                }
            }
            //System.out.print(i+"\t");
        }
        return bestMove;
    }
}
\end{lstlisting}

\begin{lstlisting}[caption=標準入力:戦略クラス内:N目並べ,label=prog32]
int prompt(String name) {
    // 標準入力から
    BufferedReader br = new BufferedReader(new InputStreamReader(System.in));
    int num = -1;
    String fmt = null, instr = null;
    do {
        fmt = String.format("\n石を置く場所 xy を指定( %s の番です ):", name);
        System.out.print(fmt);
        try {
            instr = br.readLine();
            char cx = instr.trim().charAt(0);
            int nx;
            if (numP(cx)) {
                nx = Integer.parseInt(String.valueOf(cx));
            } else {
                nx = fromA(cx) + 10;
            }
            char cy = instr.trim().charAt(1);
            int ny;
            if (numP(cy)) {
                ny = Integer.parseInt(String.valueOf(cy));
            } else {
                ny = fromA(cy) + 10;
            }
            num = nx * WIDTH + ny;
        } catch (Exception e) {
            e.printStackTrace();
        }
        if (!(0 <= num && num < WIDTH * WIDTH)) {
            System.out.println("やり直し! 場所を xy で指定して )");
            continue;
        }
        break;
    } while (true);
    return num;
}
\end{lstlisting}


%\section*{謝辞}
%\addcontentsline{toc}{chapter}{謝辞}
%
\begin{thebibliography}{99}
  \bibitem{1}
\end{thebibliography}
%
% END DOCUMENT
\end{document}
%
